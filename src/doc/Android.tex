1,计算图片大小:
图片内存大小的计算公式是 :图片高度 * 图片宽度 * 一个像素占用的字节数
numBytes = width * height * bitsPerPixel / 8
4个byte来表示--1个byte(8位)代表red,另外3个byte分别代表green、blue和alpha透明通道。这个就简称RGBA8888

=========================================
2,APP 启动流程
ActivityManagerService:(AMS)AMS是Android中最核心的服务之一,主要负责系统中四大组件的启动、切换、调度及应用进程的管理和调度等工作,
                         其职责与操作系统中的进程管理和调度模块相类似,因此它在Android中非常重要,它本身也是一个Binder的实现类。

Instrumentation:监控应用程序和系统的交互;

ActivityThread:应用的入口类,通过调用main方法,开启消息循环队列。ActivityThread所在的线程被称为主线程;

ApplicationThread:ApplicationThread提供Binder通讯接口,AMS则通过代理调用此App进程的本地方法

ActivityManagerProxy:AMS服务在当前进程的代理类,负责与AMS通信。

ApplicationThreadProxy:ApplicationThread在AMS服务中的代理类,负责与ApplicationThread通信。

可以说,启动的流程就是通过这六个大类在这三个进程之间不断通信的过程;

我先简单的梳理一下app的启动的步骤:

(1)启动的起点发生在Launcher活动中,启动一个app说简单点就是启动一个Activity,那么我们说过所有组件的启动,切换,调度都由AMS来负责的,
    所以第一步就是Launcher响应了用户的点击事件,然后通知AMS

(2)AMS得到Launcher的通知,就需要响应这个通知,主要就是新建一个Task去准备启动Activity,并且告诉Launcher你可以休息了(Paused);

(3)Launcher得到AMS让自己“休息”的消息,那么就直接挂起,并告诉AMS我已经Paused了;

(4)AMS知道了Launcher已经挂起之后,就可以放心的为新的Activity准备启动工作了,
    首先,APP肯定需要一个新的进程去进行运行,所以需要创建一个新进程,这个过程是需要Zygote参与的,
    AMS通过Socket去和Zygote协商,如果需要创建进程,那么就会fork自身,创建一个线程,新的进程会导入ActivityThread类,
    这就是每一个应用程序都有一个ActivityThread与之对应的原因;

(5)进程创建好了,通过调用上述的ActivityThread的main方法,这是应用程序的入口,在这里开启消息循环队列,这也是主线程默认绑定Looper的原因;

(6)这时候,App还没有启动完,要永远记住,四大组建的启动都需要AMS去启动,将上述的应用进程信息注册到AMS中,AMS再在堆栈顶部取得要启动的Activity,
    通过一系列链式调用去完成App启动;


=========================================

3,Android 系统启动过程
  从系统层看:

  linux 系统层
  Android系统服务层
  Zygote
  从开机启动到Home Launcher:

  启动bootloader (小程序;初始化硬件)
  加载系统内核 (先进入实模式代码在进入保护模式代码)
  启动init进程(用户级进程 ,进程号为1)
  启动Zygote进程(初始化Dalvik VM等)
  启动Runtime进程
  启动本地服务(system service)
  启动 HomeLauncher

=========================================

4,APk 安装过程
  Android应用安装有如下四种方式:
  1.系统应用安装――开机时完成,没有安装界面

  2.网络下载应用安装――通过market应用完成,没有安装界面

  3.ADB工具安装――没有安装界面。

  4.第三方应用安装――通过SD卡里的APK文件安装,有安装界面,由 packageinstaller.apk应用处理安装及卸载过程的界面。

  应用安装的流程及路径
  应用安装涉及到如下几个目录:

  system/app ---------------系统自带的应用程序,获得adb root权限才能删除

  data/app ---------------用户程序安装的目录。安装时把 apk文件复制到此目录

  data/data ---------------存放应用程序的数据

  data/dalvik-cache----将apk中的dex文件安装到dalvik-cache目录下(dex文件是dalvik虚拟机的可执行文件,其大小约为原始apk文件大小的四分之一)

  安装过程:
  复制APK安装包到data/app目录下,解压并扫描安装包,把dex文件(Dalvik字节码)保存到dalvik-cache目录,并data/data目录下创建对应的应用数据目录。

=========================================

5,Activity启动涉及到的类
首先要简单介绍一下Activity启动过程涉及到的类,以便于更好的理解这个启动过程。

ActivityThread:App启动的入口
ApplicationThread:ActivityThread的内部类,继承Binder,可以进程跨进程通信。
ApplicationThreadProxy:ApplicationThread的一个本地代理,其它的client端通过这个对象调用server端ApplicationThread中方法。
Instrumentation:负责发起Activity的启动、并具体负责Activity的创建以及Activity生命周期的回调。一个应用进程只会有一个Instrumentation对象,
                 App内的所有Activity都持有该对象的引用。
ActivityManagerService:简称AMS,是service端对象,负责管理系统中所有的Activity
ActivityManagerProxy:是ActivityManagerService的本地代理
ActivityStack:Activity在AMS的栈管理,用来记录已经启动的Activity的先后关系,状态信息等。通过ActivityStack决定是否需要启动新的进程。
ActivityRecord:ActivityStack的管理对象,每个Activity在AMS对应一个ActivityRecord,来记录Activity的状态以及其他的管理信息。
                其实就是服务器端的Activity对象的映像。
TaskRecord:AMS抽象出来的一个“任务”的概念,是记录ActivityRecord的栈,
            一个“Task”包含若干个ActivityRecord。AMS用TaskRecord确保Activity启动和退出的顺序。


Activity 启动过程
Activity 启动过程是由 ActivityMangerService(amS) 来启动的,
底层原理是 Binder实现的 最终交给 ActivityThread 的 performActivity 方法来启动她

ActivityThread大概可以分为以下五个步骤

通过ActivityClientRecoed对象获取Activity的组件信息
通过Instrument的newActivity使用类加载器创建Activity对象
检验Application是否存在,不存在的话,创建一个,保证 只有一个Application
通过ContextImpl和Activity的attach方法来完成一些初始化操作
调用oncreat方法

=========================================

6,view 加载流程
1.通过Activity的setContentView方法间接调用Phonewindow的setContentView(),
  在PhoneWindow中通过getLayoutInflate()得到LayoutInflate对象

2.通过LayoutInflate对象去加载View,主要步骤是

(1)通过xml的Pull方式去解析xml布局文件,获取xml信息,并保存缓存信息,因为这些数据是静态不变的

(2)根据xml的tag标签通过反射创建View逐层构建View

(3)递归构建其中的子View,并将子View添加到父ViewGroup中

=========================================
7,View的绘制流程
  这一部分打算从四个方面来说:

  1.View树的绘制流程

  2.mesure()方法

  3.layout()方法

  4.draw()方法

  首先说说这个View树的绘制流程:

  说到这个流程,我们就必须先搞清楚这个流程是谁去负责的

  实际上,view树的绘制流程是通过ViewRoot去负责绘制的,ViewRoot这个类的命名有点坑,最初看到这个名字,
  翻译过来是view的根节点,但是事实完全不是这样,ViewRoot其实不是View的根节点,它连view节点都算不上,
  它的主要作用是View树的管理者,负责将DecorView和PhoneWindow“组合”起来,而View树的根节点严格意义上来说只有DecorView;
  每个DecorView都有一个ViewRoot与之关联,这种关联关系是由WindowManager去进行管理的;

  那么decorView与ViewRoot的关联关系是在什么时候建立的呢?答案是Activity启动时,
  ActivityThread.handleResumeActivity()方法中建立了它们两者的关联关系,
  当建立好了decorView与ViewRoot的关联后,ViewRoot类的requestLayout()方法会被调用,
  以完成应用程序用户界面的初次布局。也就是说,当Activity获取到了用户的触摸焦点时,就会请求开始绘制布局,
  这也是整个流程的起点;而实际被调用的是ViewRootImpl类的requestLayout()方法



 =========================================

8.Android内存管理:

(1)分配机制:弹性分配,刚开始会为APP分配小额内存,根据每个APP的物理内存大小分配,然后在运行时,弹性的为其分配大小;

(2)回收机制:五大分级,前台->可见->服务->后台->空进程,优先级越低,被杀死的概率越大,lru算法,回收效益;



 =========================================
8,Service 生命周期
 =========================================
9,IntentService
 =========================================
10,bind   IdleHandler

 =========================================
11,flutter 与native通信
1.MethodChannel:Flutter端向native端发送通知,通常用来调用native的某一个方法。

2.EventChannel:用于数据流的通信,有监听功能,比如电量变化后直接推送给Flutter端。

3.BasicMessageChannel:用于传递字符串或半结构体的数据。

register plugin


 =========================================
12, okhttp 拦截器
     1>:RetryAndFollowUpInterceptor:重试拦截器
     2>:BridgeInterceptor:基础的拦截器
     3>:CacheInterceptor:缓存拦截器
     4>:ConnectInterceptor:连接的拦截器
     5>:CallServerInterceptor:





=========================================

13 线程池
提交一个任务,线程池里存活的核心线程数小于线程数corePoolSize时,线程池会创建一个核心线程去处理提交的任务。
如果线程池核心线程数已满,即线程数已经等于corePoolSize,一个新提交的任务,会被放进任务队列workQueue排队等待执行。
当线程池里面存活的线程数已经等于corePoolSize了,并且任务队列workQueue也满,判断线程数是否达到maximumPoolSize,即最大线程数是否已满,如果没到达,创建一个非核心线程执行提交的任务。
如果当前的线程数达到了maximumPoolSize,还有新的任务过来的话,直接采用拒绝策略处理。
四种拒绝策略
AbortPolicy(抛出一个异常,默认的)
DiscardPolicy(直接丢弃任务)
DiscardOldestPolicy(丢弃队列里最老的任务,将当前这个任务继续提交给线程池)
CallerRunsPolicy(交给线程池调用所在的线程进行处理)

=========================================
14,团队
1,明确需求
2,拆分需求,工作拆分
3,技术栈确定,总纲领确定
4,代码质量监控,项目维护

人才培养,团队建设
1,功能完成(基本)
2,重构,完善,修复,提高
3,技术分享,新技术探究

=========================================


15,优先级由高到低排列

前台进程
可见进程
服务进程
后台进程
空进程

=========================================